\section{Diskussion}
Et af de krav der blev stilt var at der skulle bruges så få globale variable som muligt. I vores program bruges der en global variabel til vores timer for at tælle antal millisekunder og nogle til vores LED, der skulle køre selvstendig. Selv om der altså bruges nogle blev det kun blev gjort i de tilfælde hvor det var hensiktsmæssigt og ikke hindrede abstraktionen, strukturen og modulariteten i vores projekt.
Under planlægningsfasen blev der planlagt at de forskellige levels skulle have forskellige baggrunde. Dette nåede vi ikke at implementere, men det er dog ikke væsentligt, da programmets kerne-funktionalitet er fungerende.\\
Grundig brugertesting af programmet hvor brugerne ikke fik nogle instrukser blev gennemført. Resultatet af testene, er at spillet er brugervenligt, men at det kan være vanskeligt for brugere at finde ud af at bruge knapperne på boardet. Det sidstnævnte omhandler ikke software og er sådan set ikke vores fokus i dette projekt. \\ 
Vi har så længe det gav mening, forsøgt at undgå globale variable. Vi har brugt mange globale konstanter, da vi mener det øger læsbarheden og gør ændringer i koden hurtigere og simplere. Vi har ydermere forsøgt at adskille de forskellige moduler, så de kun kommunikerer indirekte igennem main metoden.