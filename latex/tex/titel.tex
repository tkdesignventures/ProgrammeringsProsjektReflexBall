\begin{titlepage}

\centering \parindent=0pt
\newcommand{\HRule}{\rule{\textwidth}{1mm}}
\vspace*{\stretch{1}} \HRule\\[0.7cm]\Huge\bfseries
30010 - Programmeringsprojekt \\[0.7cm] % Kursusnummer og navn
Vending Machine Project\\ % Titel
\HRule\\[2cm]  
\Large
Group 3
\\
\large

Tore Gederaas Kanstad, s144021 \\

Martin Boye Brunsgaard, s144012	\\

Peter Asbjørn Leer Bysted, s144045 \\

\vspace*{\stretch{1}} \normalsize

Alle medlemmer har været tilstede under øvelserne, og deltaget i udarbejdelse af journalerne. Ydermere har arbejdet været fordelt ligeligt over gruppemedlemerne, og løst i fællesskab. Rapport
\vspace*{\stretch{1}}
\begin{flushleft}
Tecnical University of Denmark DTU\\ % Uddannelsesinstitusion
National Space Institute\\ 
30010 - Programming Project\\ % Kursusnummer og navn
17.05.2015 %Måned og år
\end{flushleft}
\end{titlepage}
\newpage
\renewcommand{\abstractname}{Abstract}
\begin{abstract}

This report covers the reflexbal game, which is a mandatory part of the B.Sc. EE course 30010 Programming Project.\\ The report documents the entire course of the exercises in which the digital logic behind a simple vending machine was designed using VHDL. The project was split into three sub-assignments: The first assignment was to drive a seven-segment display with hexadecimal numbers, the second was to display two different 2-digit decimal numbers simultaneously  and the last was to implement a CPU using a data path controlled by a FSM. The 3 assignements were combined into one circuit and implemented on a Basys2 Spartan FPGA board.
\end{abstract}
\newpage
\tableofcontents


\newpage