\begin{titlepage}

\centering \parindent=0pt
\newcommand{\HRule}{\rule{\textwidth}{1mm}}
\vspace*{\stretch{1}} \HRule\\[0.7cm]\Huge\bfseries
30010 - Programmeringsprojekt \\[0.7cm] % Kursusnummer og navn
Reflexball\\ % Titel
\HRule\\[2cm]  
\Large
Gruppe 3
\\
\large
Martin Boye Brunsgaard, s144012(1)	\\
Tore Gederaas Kanstad, s144021(2) \\
Peter Asbjørn Leer Bysted, s144045(3) \\
\begin{figure}[h]
\begin{center}
\includegraphics[scale=0.6]{img/faces.jpg}
\end{center}
\end{figure}

\vspace*{\stretch{1}} \normalsize

Alle medlemmer har været tilstede under øvelserne, og deltaget i udarbejdelse af journalerne. Ydermere har arbejdet været fordelt ligeligt over gruppemedlemerne, og løst i fællesskab. Rapporten er blevet udarbejdet og gennemlæst i kollektiv.
\vspace*{\stretch{1}}
\begin{flushleft}
Tecnical University of Denmark DTU\\ % Uddannelsesinstitusion
National Space Institute\\ 
30010 - Programming Project\\ % Kursusnummer og navn
25.06.2015 %Måned og år
\end{flushleft}
\end{titlepage}
\newpage
\renewcommand{\abstractname}{Abstract}
\begin{abstract}

This report is a mandatory part of the B.Sc. EE course 30010, Programming Project.\\ The report documents the entirety of this course, including the journals and the final product, Reflexball, a program written in C. \\ The first four days the authors learned how to use the ZDS II - Z8Encore! 4.9.3 tools, how to access the timers, the LED's, the buttons and how to use the terminal for displaying graphic. The code from these exercises were used to implement the HAL and some of the API for the project. The project was successfully implemented on a Z8 Encore Evaluation Board.
\end{abstract}
\newpage
\tableofcontents


\newpage