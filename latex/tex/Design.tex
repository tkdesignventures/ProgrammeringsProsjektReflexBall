\section{Design af Reflexball}
I udarbejdelsen af dette program havde vi nogle forskellige tekniske krav og mål, som vi ønskede at designe programmet efter.
\subsection{Tekniske mål}
Vi lavede en liste af krav til programmets design som vi i så høj grad som muligt ønskede at overholde. 
\begin{enumerate}
\item Vi ønsker en veldefineret struktur. Vi vil derfor undgå globale variable i så høj grad som muligt, derfor skal vi lave funktioner som tager pegere til strukturer eller variable som inputs, frem for at tilgå globale variable. Få undtagelser findes dog til dette, f.eks. i modul der tilgår timeren.
\item Vi ville udvikle nogle moduler  der var uafhængige af hinanden, således at vores grafik i mindst mulig grad kommunikerede med vores modul indeholdende spillets back-end(Refball.c). Denne kommunikation skal foregå igennem main-metoden, således man let kan få et overblik ved at se på main-metoden.
	
\end{enumerate}


\subsection{Krav til spillet}
\subsubsection{Overordnede krav til spillet}
\begin{enumerate}
\item Spillet er et arkanoid spil, bestående af 5 levels. Banerne skal være i stigende sværhedsgrad. Dette gøres ved at boksene gøres stærkere, således de skal rammes flere gange for at ødelægges, og også tilføje flere kasser.
\item Der skal være mulighed for at vælge sværhedsgrad, hvilket afgør hvor mange liv spilleren har, og hvor hurtigt bolden bevæger sig.
\item Hvis spilleren ikke har flere liv tilbage, afsluttes spillet og der vises game over på skærmen. Efter et par sekunder går spillet automatisk tilbage  til menuen.
\item Hvis spilleren vinder spillet vises et victory-screen og efter et par sekunder går spillet automatisk tilbage til menuen.
\item Når banen begynder, eller hvis spilleren mister et liv, placeres bolden over strikeren, og spilleren kan frit bevæge strikeren, hvor bolden følger efter. Hvis spilleren trykker på den givne knap, affyres bolden.
\item Spillerens liv og power skal skrives på LED-skærmen når spillet er igang
\item Spilleren samler power hver gang han ødelægger en kasse. Hvis brugeren trykker på venstre og højre-tasten på en gang bruger han sit power og akitverer high power. Når high power er aktiveret ødelægges kasser når de rammes, uafhængigt af deres liv, og bolden reflekteres ikke, men fortsætter gennem kassen. Power fratrækkes 1 hver gang den ødelægger en kasse.
\item Når spilleren bruger high power, vinder en bane, vinder spillet eller dør skal der rulles en tekst over LED-skærmene. Alt afhænigt af situationen, skal livene og tiden igen vises på skærmen efter teksten er rullet over.
\item Hver gang spilleren går videre til næste level, får spilleren fuldt liv og spillerens power sættes til 0.
\item Hver gang spilleren går videre til næste level, får spilleren fuldt liv og spillerens power sættes til 0.
\item Spillere kan pause spillet, ved at trykke på en af knapperne.
\item Ved at trykke på alle knapper samtidigt, aktiverer man chef-mode, som giver en blank skærm.  
\item Der er ikke noget point-system i spillet, da vi vælger at lægge fokus andre steder.
\end{enumerate}
\subsubsection{Krav til strikeren}
\begin{enumerate}
\item Strikeren skal maskimalt fylde 10\% af skærmen på x-aksen. 
\item  Strikeren skal være delt ind i 3 forskellige områder. Disse 3 områder skal reflektere bolden på forskellig vis afhængig af indgangsvinklen og hvilken del af strikeren den rammer. Reflektionen skal findes igennem trial and error, og vurderes hvad der virker mest naturligt. I oplægget var der lagt op til at strikeren skulle have 5 områder, men vi syntes ikke det fungerede særligt godt, da det var vanskeligt for brugeren at forholde sig til. Vi har derfor valgt 3 områder.
\item Brugeren skal kunne styre strikeren, vha. knapperne på boardet.
\end{enumerate}
\subsubsection{Krav til bolden}
\label{Ballkrav}
\begin{enumerate}
\item Bolden skal have et x- og y koordinat og en retningsvektor, begge i 18.14 format. Bolden har desuden nogle variable med info om spillerens power, om bolden er ude og om spilleren har aktiveret power.
\item Boldens retningsvektor skal altid have længden 1, da dette gør kollisionstest let.

\end{enumerate}
\subsubsection{Krav til boksene}
\begin{enumerate}
\item Alle bokse skal have de samme dimensioner, vi valgte 2x6 pixels.
\item Boksene skal kunne have forskellig styrke, således at nogle kasser skal rammes flere gange før de går i stykker. Kassens styrke skal således repræsenteres ved en farve, og farven ændrer sig således også når man rammer en kasse uden at ødelægge den.
\item Hvis man rammer boksen på den horizontale side, skal y-elementet af retningsvektoren inverteres. 
\item Hvis man rammer boksen på den vertikale side, skal x-elementet af retningsvektoren inverteres.
\item Hvis man rammer et hjørne, skal både x- og y-elementet inverteres.
\item Når en kasse bliver ødelagt slettes den fra banen
\end{enumerate}


\subsection{Timere}
På Z8 Encore Evauluation Boardet er der 4 forskellige timere, timer0 til timer3. Disse timere kan konfigureres efter brugerens behov. I vores projekt har vi brugt 2 timere, en til at styre spillets tid, og en anden til at styre LED skærmene. Disse 2 timere er hhv. timer0 og timer1.
\subsubsection{Timer0}
Timer0 er en timer der sender et tick hvert millisekund. Timeren bliver brugt i main-funktionen og til debouncing af knapperne. Timeren er sat i continous mode, da vi ønsker at den blot skal fortsætte ubetinget, og der foretages ingen clock division af tælleren. Reload værdien fandtes ved udregningen i ligning \ref{timer0}. Interrupt Prioriteten sættes til høj ved at skrive $0x20$ til både IRQ0ENH og IRQ0ENL.
\begin{equation}
\label{timer0}
Reloadvalue = 0.001 \si{s} \cdot 18.432.000 \si{s^{-1}} = 4800_{16} 
\end{equation}
\subsubsection{Timer1}
Timer1 er en timer der sender et tick hvert 500 $\mu$s. Timeren bliver kun brugt i \textbf{led.h}. Denne timer er også sat i continous mode, og der bliver heller ikke her foretaget clock division. Reload værdien fandtes ved udregningen i ligning \ref{timer1}. Interrupt prioriten sættes til lav ved kun at skrive til IRQ0ENL.

\begin{equation}
\label{timer1}
Reloadvalue  = 0.005 \si{s} \cdot 18.432.000 \si{s^{-1}} = 2400_{16}  
\end{equation}