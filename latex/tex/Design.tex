\section{Design af Reflexball}
I udarbejdelsen af dette program havde vi nogle forskellige tekniske krav og mål, som vi ønskede at designe programmet efter.
\subsection{Tekniske mål}
Vi lavede en liste af krav til programmets design som vi i så høj grad som muligt ønskede at overholde. 
\begin{enumerate}
\item Vi ønsker en veldefineret struktur. Vi vil derfor undgå globale variable i så høj grad som muligt, derfor skal vi lave funktioner som tager pegere til strukturer eller variable som inputs, frem for at tilgå globale variable. Få undtagelser findes dog til dette, f.eks. i bibloteket der tilgår timeren.
\item Vi ville udvikle nogle biblioteker  der var uafhængige af hinanden, således at vores grafik i mindst mulig grad kommunikerede med vores bibliotek indeholdende spillets back-end(Refball.c). Denne kommunikation skulle såfremt foregå igennem main-metoden, således man let kan få et overblik ved at se på main-metoden.
	
\end{enumerate}


\subsection{Krav til spillet}
\subsubsection{Overordnede krav til spillet}
\begin{enumerate}
\item Spillet er et arkanoid spil, bestående af 3 levels. Banerne skal være i stigende sværhedsgrad. Dette gøres ved at boksene gøres stærkere, således de skal rammes flere gange for at ødelægges, og også tilføje flere kasser.
\item Spilleren har 3 liv til at starte med, og får et liv for hver bane han vinder.
\item Hvis spilleren ikke har flere liv tilbage, afsluttes spillet og der vises game over på skærmen. Efter et par sekunder går spillet automatisk tilbage  til menuen.
\item Når banen begynder, eller hvis spilleren mister et liv, placeres bolden over strikeren, og spilleren kan frit bevæge strikeren, hvor bolden følger efter. Hvis spilleren trykker på den givne knap, affyres bolden i en opadgående lodret linje. 
\item Spillerens liv og tiden skal skrives på LED-skærmen når spillet er igang
\item Spilleren samler power hver gang han rammer en kasse. Hvis brugeren trykker på venstre og højre-tasten på en gang bruger han sit power og akitverer hiii power. Når hiii power er aktiveret ødelægges kasser når de rammes, uafhængigt af deres liv, og bolden reflekteres ikke, men fortsætter gennem kassen.
\item Når spilleren bruger hiii power, vinder en bane, vinder spillet eller dør skal der rulles en tekst over LED-skærmene. Alt afhænigt af situationen, skal livene og tiden igen vises på skærmen efter teksten er rullet over.
\end{enumerate}
\subsubsection{Krav til strikeren}
\begin{enumerate}
\item Strikeren skal maskimalt fylde 10\% af skærmen på x-aksen. 
\item  Strikeren skal være delt ind i 5 forskellige områder. Disse 5 områder skal reflektere bolden på forskellig vis afhængig af indgangsvinklen og hvilken del af strikeren den rammer. 
\item Brugeren skal kunne styre strikeren, vha. knapperne på boardet.
\end{enumerate}
\label{Ballkrav}
\subsubsection{Krav til bolden}
\begin{enumerate}
\item Bolden skal implementeres som en struct, bestånde af et x- og y koordinat og en retningsvektor i 18.14 format. Bolden har desuden nogle variable med info om spillerens power, om bolden er ude og om spilleren har aktiveret power.
\item Boldens retningsvektor skal altid have længden 1, da dette gør kollisionstest let.
\end{enumerate}
\subsubsection{Krav til boksene}
\begin{enumerate}
\item Alle bokse skal have de samme dimensioner, vi valgte 2x6 pixels.
\item Boksene skal kunne have forskellig styrke, således at nogle kasser skal rammes flere gange før de går i stykker. Kassens styrke skal således repræsenteres ved en farve, og farven ændrer sig således også når man rammer en kasse uden at ødelægge den.
\item Hvis man rammer boksen på den horizontale side, skal y-elementet af retningsvektoren inverteres. 
\item Hvis man rammer boksen på den vertikale side, skal x-elementet af retningsvektoren inverteres.
\item Hvis man rammer et hjørne, skal både x- og y-elementet inverteres.
\item Når en kasse bliver ødelagt slettes den fra banen
\end{enumerate}



\subsection{}