\subsection{Design af Reflexball}
I designet af Reflexball blev der givet forskellige krav til udførelsen af spillet.

\subsubsection{Krav til strikeren}
Der blev stillet nogle krav til strikeren:
\begin{enumerate}
\item Strikeren skal maskimalt fylde 10$%$ af skærmen på x-aksen. 
\item  Strikeren skal være delt ind i 5 forskellige områder. Disse 5 områder skal reflektere bolden på forskellig vis afhængig af indgangsvinklen og hvilken del af strikeren den rammer. 
\item Brugeren skal kunne styre strikeren, vha. knapperne på boardet.
\end{enumerate}

Desuden blev der stillet krav til spillets features, vi valgte at implementere det således:
\begin{enumerate}
\item Spillet er et arkanoid spil, bestående af 3 levels. Banerne skal være i stigende sværhedsgrad.
\item Spilleren har 3 liv til at starte med, og får et liv for hver bane han vinder.
\item Hvis spilleren ikke har flere liv tilbage, afsluttes spillet og der vises game over på skærmen. Brugeren har mulighed for at gå tilbage til menuen og starte forfra igen.
\item Når banen begynder, eller hvis spilleren mister et liv, placeres bolden over strikeren, og spilleren kan frit bevæge strikeren, hvor bolden følger efter. Hvis spilleren trykker på den givne knap, affyres bolden i en opadgående lodret linje. 
\end{enumerate}

