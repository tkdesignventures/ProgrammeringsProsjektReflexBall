\section{Implementation}
Vi vil i denne sektion gennemgå implementationen af spillet. Vi har valgt blot at gennemgå main-metoden, og flowet af denne. Info om de andre moduler kan findes i dokumentationen.
\subsection{main.c}
Det her spil styres af en main.c fil med en main funktion. For at forbedre strukturen og øge læsbarheden er selve gameplayet håndteret af en funktion der hedder \textbf{Game}. Denne funktion bliver kaldet af main når brugeren vælger start game og returnerer antal liv der er tilbage når spillet afsluttes. På denne måde registeres sejr /  tab. \\
Main funktionen starter med at tegne menuen, sende teksten ”Welcome” til LED-displayet og går derefter ind i en uendelig løkke, hvor den venter på at brugeren trykker  på en knap. Når brugeren trykker på den venstre  og  midterste knap bladrer man igennem menuen ved at øge / formindske \textit{selectedOption}, der holder styr på hvor man befinder sig i menuen. Når brugeren trykker på højre knap undersøger programmet \textit{selectedOptions} værdi og foretager en handling baseret på dens værdi. Det kan være at starte et nyt spil, ændre sværhedssgrad eller vise hjælp.\\
Når et nyt spil begynder, starter funktionen Game med at positionere strikeren, vælge hvor mange bolde brugeren skal have hver level, hvor hurtigt bolden skal køre og tegner tilsidst banen. \\ Derefter går man ind i en løkke der kører en gang per level helt til max level er nået eller til brugeren ikke har flere liv. Hver gang en ny level starter får brugeren fuldt liv og bolden bliver placeret over strikeren. LED-displayet viser også hvilken level man er nået til. For at skrive tal fra variable på LED-displayet type-castes de til den tilhørende char-værdi og lægges ind i et char array der bliver konkateneret med den resterende streng. Denne streng sendes  til LED-displayet med funktionen \textbf{LEDSetString}. Funktionen \textbf{setLedMode} benyttes for at rigtig visning bliver brugt.\\
Bolden tegnes med funktionen \textbf{drawChar}, hvor det 3. argument er typen character der skal tegnes. Oftest er dette et ”o”, men hvis bolden rammer strikeren eller kanterne printes  der en char der gør at disse bliver grafisk bevaret. Hvilken character der skal styres af funktionen \textbf{checkBall}. Envidere dannes og tegnes bokserne med farver der svarer til styrken.\\
Når initialiseringen for en level er færdig går man ind i en ny løkke, som kører så længe man har liv og bokse tilbage. Til at begynde med registrerer programmet hvilke knapper der et trykket ind. Når brugeren trykker højre knap skydes bolden.  Hvis brugeren trykker på den højre tast igen pauses spillet. når brugeren trykker alle tre knapper bliver skærmen renset (chef-mode).\\
Hvis spillet ikke er pauset har brugeren mulighed for at flytte strikeren med venstre og midterste knap, og aktivere High Power med begge knapper. Når High Power aktiveres ruller teksten "Power!" over LED-displayet og BEL-characteren printes(siger en lyd, hvis PuTTY er indstillet korrekt)\\
Det benyttes en tæller til at regulere den frekvens boldens position opdateres ved. Hvis bolden er skudt ud og aktiv, checkes bolden for om den kommer til at ramme en kant, en boks eller strikeren. Hvis den kommer til at gøre det endres boldens retning. Envidere males bolden over.\\
Der testes for om bolden er ude af banen. Hvis det er tilfældet bliver bolden sat over strikeren igen og brugeren har nu en mindre bold tilbage. LED-displayet opdateres med rigtige antal bolde og mængde \textit{Power}. Til sidst flyttes og tegnes bolden. Bolden tegnes med farven rød hvis High Power er aktiveret. \\
Når brugeren ikke har flere bolde  tilbage eller gennemført spillet kaldes funktionen \textbf{drawGameOver} eller \textbf{drawVictory}. De funktioner bruger lang tid på at køre, hvilket giver brugeren tid til at se hvad der står.
